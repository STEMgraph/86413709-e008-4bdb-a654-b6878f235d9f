\learningobjective{At the end of this challenge, the scholar will be able to use the echo, ls and cat commands. And perform simple redirections.}
\begin{challenge}
    \chatitle{Input and Output to and from the command line interpreter} 
 
    \begin{chadescription}
        When we are talking about computers, what we mean is a piece of electronic hardware, that is able to perform algorithmic and logic mappings from a binary number to another binary number.
        Interfacing with this hardware can be tricky, thats why modern computers come with an operating system.
        This operating system provides us with a helper program, called the Shell.
        The Shell is a program, that \textit{wraps} around the operating-systems inner core, also known as the kernel.
        Sending commands to the kernel can be really tedious and error-prone, thats why the Shell gives us a more user-friendly interface.
        The Shell itself is the program that interpretes our input from RAM, prepares the data for the kernel sends a request to have it evaluated and stores the hardwares answer back in RAM.
        Before there were interactive keyboard terminals to computers, the content of the RAM was set manually, by connecting switches and wires to high or low voltages.
        In the 1950, the \textbf{MIT Whirlwind} computer was the first computer to use an electronic typewriter, to interact with the user in real-time - that means in cycles of \textit{read-compute-write} while the computer was running.
        An electric typewriter is a magnificent machine. 
        In contrast to mechnical typewriters that have a mechnical connection between each key and a specific hammer, keys in electrical typewriters activate a set of switches to electrically choose which character shall be printed to the paper and sort of a motor to hammer.
        The idea of the interactive terminal was it, to interrupt the signal from the key to the typing-electronics and rerouting it to a computer, that performs a transformation of the electrical signal, and sends the modified signal to the typing-electronics. 
        A button that was pressed on the keyboard was translated into a signal that was sent to the computers memory using a serializer-deserializer-circuit.
        The computer would perform a transformation of that piece of memory and generate a new sequence of electrical signals.
        An output that was written by the shell-program to memory was translated back into a signal that was sent to the terminals electrical typing-system.
        The structure of todays machines is still somewhat comperable, but instead of an electric-typewriter-terminal, we now have computers with a so-called terminal-emulator. 
        You should always keep in mind, that the calculation mechanism of the computer and the input and output are seperate systems!\\
        In 1988 a group of electrical engineers, mathematicians and physicists came up with the idea of writing down a standardized way how input, output and operating-systems should interact with one another, making it easier for programmers, to build programs for different types of computers. 
        You have to imagine, that before that this standardization, every computer was basically completely different. 
        The standard is still widely used in an updated version and is called Portable Operating System Interface or short: \textbf{POSIX}.
        POSIX also defines what a \textit{FILE} is supposed to be and how it should be used. 
        You probably know some FILEs from your home-computer already but you may have not heard of the kind of FILEs that we will be talking about in this challenge.\\
        Operating-systems come with a piece of software called a \textit{filesystem}.
        It is the responsibility of the filesystem, to give every piece of memory on the computer, that is available to the programmer a human-readable name.
        Please keep in mind, that our computer is just wires, capacitors, transistors and such.
        There are no real names for our harddisk, our RAM or anything else. 
        Thus the filesystem is configured in a way, that we can address these memory-regions by an alias or identifier. 
        These identifiers are called filenames. 
        Let's explore some basic commands of the command-line-interpreter, that we can use to interact with the filesystem!
    \end{chadescription}

    \begin{task}
        Start up your terminal-emulator, type \texttt{echo "Hallo"} and press \texttt{Enter}.
        The Terminal will respond with \texttt{Hallo}.
        \texttt{echo} is a subroutine, that is build into the command-line-interpreter.
        Everytime a program or programmer calls the \texttt{echo}-routine, the command-line-interpreter will write the argument to the standard output.\\
        Now call \texttt{ls} and press \texttt{Enter}.
        \texttt{ls} is a program that is installed on your computer.
        It generates a list of files, that are available to you at the moment and echos it to the standard output.
        If you have not used the terminal that much before, you might see only a few or no files at all. 
        If you have used your computer and especially the terminal a lot, you might see more files there. 
        Let's start by writing some text to our filesystem by typing:
        \begin{lstlisting}
            echo 'My First Text' > new_file.txt 
        \end{lstlisting}
        Press \texttt{Enter} to confirm.
        Let's use \texttt{ls} again and see what we get.

        \begin{quesitons}
            \item What does the command \texttt{echo} do?
            \item What does the command \texttt{ls} do?
            \item What do you observe as output, when using \texttt{echo} with a postfix \texttt{>}?
            \item How did the output of \texttt{ls} change after using \texttt{echo} with a postfix \texttt{>}?
        \end{quesitons}
    \end{task}

    \begin{task}
        As you have probably guessed correctly, the command we used generated a new file on our computer.
        THe command has two seperate parts, the \texttt{echo} and the \textit{redirection}.
        We discussed the echo already, and found out that it echos the argument to the standard output.
        The \texttt{>} is called the \textit{redirection operator}.
        It tells the command-line-interpreter, that the result of the evaluation of the expression to the left should not be written to the standard output, but to a different memory-location, usually indicated by a filename: in this case \texttt{new_file.txt}.
        Let's inspect the new file now.
        Type \texttt{cat new_file.txt} and press \texttt{Enter}.
        \texttt{cat} is another program installed on your computer.
        It reads the content of the argument and echos it to the standard output.
        You should see a pattern here already: many of the tools we use from the command-line, are executed by calling their name and passing an argument.
        Some programs also accept multiple arguments.\\
        Create a second file now, called \texttt{new_file2.txt} and write some text into it, using \texttt{ls >}.
        Then use \texttt{cat} to inspect the content of the new file.
        \texttt{cat} can also be called with two arguments. 
        Call \texttt{cat new_file.txt new_file2.txt} and inspect the output.
        You see your Output on the standard terminal output again, redirect the output to a new file, called \texttt{new_file3.txt} and inspect the content of the new file.
        Now, call \texttt{echo 'Hello World' > new_file3.txt} and inspect the content of the new file again.
        In a second attempt call \texttt{echo "A next line" >> new_file3.txt} and inspect the content of the new file again.
        And for the final trick, try the following command:
        \begin{lstlisting}
            (cat file1.txt file2.txt file3.txt; echo 'Another Line') > file4.txt
        \end{lstlisting}
        \begin{questions}
            \item What does the command \texttt{cat} do if you pass one parameter?
            \item What does the command \texttt{cat} do if you pass multiple parameters?
            \item What is the difference between the \texttt{>} and the \texttt{>>} operators?
            \item What happens when you put parentheses around two commands, separated by a semicolon?
        \end{questions}
    \end{task}

    \begin{task}
        You can also use your self-defined functions to generate data and store it in a file.
        Let's create a function called \texttt{summation} that adds two numbers and echos the result:
        \begin{lstlisting}
            summation() {
                echo $(( $1 + $2 ))
            }
        \end{lstlisting}
        Now let's call the function with two arguments and redirect the output to a new file:
        \begin{lstlisting}
            summation 1 2 > result.txt
        \end{lstlisting}
        \begin{questions}
            \item Inspect the content of the new file by running \texttt{cat result.txt}. Does it contain what you expected?
            \item Now call the function and replace one argument with the content of the result file like this: \texttt{summation \$(cat result.txt) 2}. What is the output and did you expect that?
            \item Run it again and redirect the output back to the result file: \texttt{summation \$(cat result.txt) \$(cat result.txt) > result.txt}. Type the \textit{UP}-arrow-key and repeat the last command. Do this a couple of times and inspect the content of the result file. What happend?
        \end{questions}
    \end{task}

    \begin{task}
        Now that you know, that you can also store data in files, let's use this to generate a file, that stores a function for you.
        We will now create a function, but instead of loading it into our memory as we did before, we will store it in a file.
        Therefore we will use \texttt{echo} command to redirect a function we wrote into a file:
        \begin{lstlisting}
            echo 'function multiply(){
                echo $(( $1 * $2 ))
            }' > multiply.sh
        \end{lstlisting}
        Make sure, that you surround the function with single quotes! 
        Everything within the single quotes will be interpreted as one argument for the \texttt{echo} command.
        \begin{questions}
            \item Inspect the content of the new file by running \texttt{cat multiply.sh}. Does it contain what you expected?
            \item Try calling the function directly by running \texttt{multiply 2 3}. What is the output and did you expect that?
            \item Now user the \texttt{source} command to load the function into your memory: \texttt{source multiply.sh}. Try calling the function again: \texttt{multiply 2 3}. What is the output and did you expect that?
            \item Close your terminal-session and open a new one. Try calling the function again: \texttt{multiply 2 3}. What is the output and did you expect that?
        \end{questions}
    \end{task}


    \begin{advice}
        You can put as many functions into one file as you like. 
        They will be stored in your file-system and you can use them whenever you want. 
        Just make sure to use the \texttt{source} command to load them into the memory of your current terminal-session.
        In one of the following challenges, you will learn how to source a file automatically whenever you open a new terminal-session.
    \end{advice}
\end{challenge}
